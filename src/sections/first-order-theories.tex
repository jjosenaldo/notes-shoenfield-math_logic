\section{First-order Theories}

\subsection{Functions and Predicates}

\begin{definition}[``Universe'']
	A \emph{``universe''} is some special set in a mathematical axiom system.
\end{definition}

\begin{definition}[Individual Functions; Individual Predicates]
	Functions from the universe to the universe are called \emph{individual functions}, and predicates
	in the universe are called \emph{individual predicates}.
\end{definition}

\subsection{Truth Functions}

\begin{definition}[Truth Value]
	Truth values are objects \textbf{T} and \textbf{F} different from each other.
\end{definition}

\begin{definition}[Truth Function]
	A \emph{truth function} is a function from the set of truth values to the set of truth values.
\end{definition}

\subsection{Variables and Quantifiers}

\begin{definition}[Individual Variables]
	\emph{Individual variables} vary on the individuals of the universe. They will be often
	referred to as just \emph{variables}.
\end{definition}

\begin{remark}
	The need for a universal quantifier can be seen when one tries to translate 
	``not all natural numbers are equal to 0'' into a formula. Also, while $x=0$ has many
	meanings, the formula $\forall x (x=0)$ has only one meaning.
\end{remark}

\begin{remark}
	If $\forall x$ or $\exists x$ is introduced in a formula with no free occurrences of $x$,
	the meaning of the formula is unchanged.
\end{remark}

\subsection{First-order Languages}

\begin{definition}[Symbols of a First-order Language; Constant; Logical and Lonlogical Symbol]
	The symbols of a first-order language consist of:
	\begin{enumerate}
		\item The variables $x$, $y$, $z$, $w$, $x'$, $y'$, $z'$, $w'$, $x''$, $\dots$.
		\item For each $n \in \nats$, the \emph{$n$-ary function symbols} and the 
		\emph{$n$-ary predicate symbols}.
		\item The symbols $\lnot$, $\lor$ and $\exists$.
	\end{enumerate}
	The \emph{equality symbol} $=$ is contained in the set of 2-ary predicate symbols.

	A \emph{constant} is a 0-ary function symbol.

	A function or predicate symbol other than $=$ is called a \emph{nonlogical symbol}. Otherwise,
	it is a \emph{logical symbol}.
\end{definition}

\begin{convention}
	The syntactical variables \synt{x}, \synt{y}, \synt{z} and \synt{w} vary through
	variables; \synt{f} and \synt{g} vary through function symbols; \synt{p} and \synt{q}
	vary through predicate symbols and \synt{e} varies through constants.
\end{convention}

\begin{definition}[Term]
	The \emph{terms} are defined by:
	\begin{enumerate}
		\item Variables are terms.
		\item If \synt{u}$_1$, \dots, \synt{u}$_n$ are terms and \synt f is a $n$-ary function
		symbol, then \synt f\synt u$_1$\dots\synt u$_n$ is a term.
	\end{enumerate}
	The syntactical variables \synt{a}, \synt{b}, \synt{c} and \synt{d} vary through
	terms.
\end{definition}

\begin{definition}[(Atomic) Formula]
	The \emph{formulas} are defined by:
	\begin{enumerate}
		\item If \synt p is $n$-ary, then \synt p\synt a$_1$\dots\synt a$_n$ is a formula.
		It is also called an \emph{atomic formula}.
		\item If \synt u is a formula, then $\lnot$\synt u is a formula.
		\item If \synt u and \synt v are formulas, then $\lor$\synt u\synt v is a formula.
		\item If \synt u is a formula, then $\exists$\synt x\synt u is a formula.
	\end{enumerate}
\end{definition}

\begin{definition}[Height of a Formula]
	The \emph{height} of a formula \synt u is the number of occurrences of $\lnot$,
	$\lor$ and $\exists$ in \synt u.
\end{definition}

\begin{definition}[First-order Language]
	A \emph{first-order language} is a language whose symbols and formulas are as
	described previously.
\end{definition}

\begin{remark}
	A first-order language is defined by its nonlogical symbols. In particular,
	if a symbol is used as a $n$-ary function (predicate) symbol in one 
	first-order language, it can only be used in other first-order languages
	as a $n$-ary function (predicate) symbol.
\end{remark}

\begin{definition}[Designator; Index of a Symbol]
	A \emph{designator} is an expression which is either a formula or a term.
	Every designator has the form \synt u\synt v$_1$\dots\synt v$_n$, where
	\synt u is a symbol, \synt v$_1$, \dots, \synt v$_n$ are designators and 
	$n$ is determined by \synt u. We call $n$ the \emph{index} of \synt u.
\end{definition}

\begin{definition}[Compatible Expressions]
	Two expressions are \emph{compatible} when one of them is a prefix
	of the other.
\end{definition}

\begin{fact}
	If \synt{uv} and \synt{u$'$v$'$} are compatible, then \synt u and \synt{u$'$} are
	compatible. If \synt{uv} and \synt{uv$'$} are compatible, then \synt v
	and \synt{v$'$} are compatible.
\end{fact}

\begin{lemma}
	If \synt u$_1$, \dots, \synt u$_n$, \synt{u'}$_1$, \dots, \synt{u'}$_n$ are designators
	and \synt u$_1$\dots\synt u$_n$ and \synt{u'}$_1$\dots\synt{u'}$_n$ are compatible,
	then each \synt u$_i$ $=$ \synt{u'}$_i$ for each $1 \le i \le n$.
\end{lemma}

\begin{theorem}[Formation Theorem]
	Every designator can be written in the form \synt u \synt v$_1$\dots\synt v$_n$, 
	where \synt u is a symbol of index $n$, and \synt v$_1$, \dots, \synt v$_n$ are designators,
	in only one way.
\end{theorem}

\begin{lemma}
	Every occurrence of a symbol in a designator \synt u begins an occurrence of a designator
	in \synt u.
\end{lemma}

\begin{theorem}[Occurrence Theorem]
	Let \synt u be a symbol of index $n$, and let \synt v$_1$, \dots, \synt v$_n$
	be designators. Then any occurrence of a designator \synt v in 
	\synt u\synt v$_1$\dots\synt v$_n$ is either all of \synt u\synt v$_1$\dots\synt v$_n$
	or a part of one of the \synt v$_i$.
\end{theorem}

\begin{definition}[Free and Bound Occurrences]
	An occurrence of \synt x in \synt A is \emph{bounded in \synt A} if it occurs in a part
	of \synt A of the form $\exists$\synt x\synt B; otherwise, it is \emph{free} on \synt A.
	We say that \synt x is \emph{free (bound) in \synt A} if some occurrence of \synt x
	is free (bound) in \synt A.
\end{definition}

\begin{fact}
	If \synt y is distinct from \synt x, then the free occurrences of \synt x in $\lnot$\synt A,
	$\lor$\synt{AB} and $\exists$\synt{yA} are just the free occurrences of \synt x in \synt A
	and \synt B.
\end{fact}

\begin{definition}[Substituibleness]
	We say that \synt a is \emph{substituible} for \synt x in \synt A when for each
	variable \synt y occurring in \synt a, no part of \synt A of the form $\exists$\synt{yB}
	contains an occurrence of \synt x which is free in \synt A.
\end{definition}

\begin{convention}
	We use \subs{\synt a}{\synt x}{\synt b} to designated the expression obtained from \synt b
	replacing each occurrence of \synt x by \synt a; and we use \subs{\synt a}{\synt x}{\synt A}
	to designate the expression obtained from \synt A by replacing each free occurrence of \synt x
	by \synt a.
\end{convention}

\begin{fact}
	The expressions \subs{\synt a}{\synt x}{\synt b} and \subs{\synt a}{\synt x}{\synt A}
	are indeed a term and a formula, respectively.
\end{fact}

% The formulas containing the connectives \&, \to and \forall and the infix versions of 
% binary connectives/formulas/predicates are abbreviations

\subsection{Structures}

\begin{definition}[Structure]
	Let $L$ be a first-order language. A \emph{structure} $\struct$ for $L$ consists of the following:
	\begin{enumerate}
		\item A nonempty set $\pipe\struct$, called the \emph{universe} of $\struct$.
		The elements of $\universe\struct$ are called the \emph{individuals} of $\struct$.
		\item For each $n$-ary function symbol \synt f of $L$, an $n$-ary function
		\synt f$_\struct$ from $\universe\struct$ to $\universe\struct$.
		\item For each $n$-ary predicate symbol \synt p of $L$ other than =,
		an $n$-ary predicate \synt p$_\struct$ in $\universe\struct$. 
	\end{enumerate}
\end{definition}

\begin{definition}[Names of Individuals]
	Let $\struct$ be a structure for $L$. For each individual $a$ of $\struct$,
	we choose a new constant, called the \emph{name} of $a$, such that each individual
	is assigned a different name. The language obtained by adding those names is 
	designated by $L(\struct)$. The syntactical variables \synt i and \synt j vary
	through names.
\end{definition}

\begin{definition}[Variable-free Expression]
	An expression is \emph{variable-free} if it contains no variables.
\end{definition}

\begin{definition}[$\struct(\msynt a)$]
	Let \synt a be a variable-free term of $L(\struct)$. If \synt is a name, then
	$\struct$(\synt a) is the individual of which \synt a is the name. Otherwise,
	\synt a must be \synt f \synt a$_1$\dots\synt a$_n$ with \synt f a function
	symbol of $L$. We then let $\struct(\msynt a)$ be 
	$\msynt f_\struct(\struct(\msynt a_1), \dots, \struct(\msynt a_n))$.
\end{definition}

\begin{remark}
	The previous definition makes sense because of the Formation Theorem.
\end{remark}

\begin{definition}[Closed Formula]
	A formuça \synt A is \emph{closed} if there is no free variable in \synt A.
\end{definition}

\begin{definition}[$\struct(\msynt A)$]
	Let \synt A be a closed formula in $L(\struct)$. If \synt A is $\msynt a = \msynt b$,
	then \[\struct(\msynt A) = \mtrue \tiff \struct(\msynt a)=\struct(\msynt b)\]

	If \synt A is $\msynt p \msynt a_1 \dots \msynt a_n$, where \synt p is not =,
	we let 
	\[\struct(\msynt A) = \mtrue \tiff \msynt p_\struct\paren{\struct(\msynt a_1),\dots,\struct(\msynt a_1)}\]
	If \synt A is $\lnot \msynt B$, then $\struct(\msynt  A)$ is
	$H_\lnot\paren{\struct(\msynt B)}$. If \synt A is $\msynt B \lor \msynt C$, then
	$\struct(\msynt  A)$ is $H_\lor\paren{\struct(\msynt B),\struct(\msynt C)}$.
	If \synt A is $\exists \msynt{xB}$, then $\struct(\msynt A) = \mtrue$ iff 
	$\struct(\msubs{\msynt i}{\msynt x}{\msynt B})=\mtrue$ for some \synt i in
	$L(\struct)$.
\end{definition}

\begin{definition}[$\struct$-instance of a Formula]
	If \synt A is a formula of $L$, an \emph{$\struct$-instance} of \synt A
	is a closed formula of the form
	$\msubs{\msynt i_1,\dots,\msynt i_n}{}{\msynt A}$ in $L(\struct)$.
\end{definition}

\begin{definition}[Valid Formula]
	A formula \synt A of $L$ is \emph{valid} in $\struct$ if $\struct(\msynt A')=\mtrue$
	for every $\struct$-instance $\msynt A'$ of \synt A.
\end{definition}

\begin{lemma}
	Let $\struct$ be a structure for $L$; \synt a a variable-free term in
	$L(\struct)$; \synt i the name of $\struct(\msynt a)$. If \synt b is a term 
	of $L(\struct)$ in which no variable except \synt x occurs, then
	$\struct(\msubs{\msynt x}{\msynt a}{\msynt b}) = \struct(\msubs{\msynt x}{\msynt i}{\msynt b})$.
	If \synt A is a formula of $L(\struct)$ in which no variable except \synt x
	is free, then
	$\struct(\msubs{\msynt x}{\msynt a}{\msynt A}) = \struct(\msubs{\msynt x}{\msynt i}{\msynt A})$.
\end{lemma}