\section{Theorems in First-Order Theories}

\subsection{The Tautology Theorem}

\begin{definition}[Elementary Formula]
	A formula is \emph{elementary} if it is either an atomic formula or an
	instantiation.
\end{definition}

\begin{definition}[Truth Valuation]
	A \emph{truth valuation} for a theory $T$ is a mapping 
	from the set of elementary formulas in $T$ to $\set{\mtrue, \mfalse}$.
	Truth valuations are extended to arbitrary formulas in the usual
	manner.
\end{definition}

\begin{definition}[Tautological Consequence]
	$\msynt B$ is a \emph{tautological consequence of} $\msynt A_1, \dots, \msynt A_n$
	if $V(\msynt B)=\mtrue$ for every truth valuation $V$ such that
	$V(\msynt A_1) = \dots = V(\msynt A_n) = \mtrue$.
\end{definition}

\begin{definition}[Tautology]
	A formula is a \emph{tautology} if it is a logical consequence of the
	empty set of formulas.
\end{definition}

\begin{fact}
	$\msynt B$ is a tautological consequence of $\msynt A_1, \dots, \msynt A_n$ iff
	$\msynt A_1 \to \dots \to \msynt A_n \to \msynt B$ is a tautology.
\end{fact}

From now on, formulas are considered to be in the set of formulas whose
logical connectives are either $\lnot$ or $\lor$.

\begin{fact}
	Let $\msynt A_1, \dots, \msynt A_n$ be formulas.
	 %whose logical connectives can only be $\lor$ or $\lnot$. 
	 It can be determined in a finite number of steps
	whether $\msynt A_1 \lor \dots \lor \msynt A_n$ is a tautology.
\end{fact}

\begin{theorem}[Tautology Theorem (Post)]
	If \synt B is a tautological consequence of \synt A$_1$, \dots, 
	\synt A$_n$ and $\thm \msynt A_1$, \dots, $\thm \msynt A_n$,
	then $\thm \msynt B$.
\end{theorem}

\begin{corollary}
	Every tautology is a theorem.
\end{corollary}

\begin{lemma}
	If $\thm \msynt A \lor \msynt B$, then $\thm \msynt B \lor \msynt A$.
\end{lemma}

\begin{fact}
	The detachment rule is admissible: %TODO: what's the right term for this?
	\begin{description}
		\item[Detachment Rule] If $\thm \msynt A \to \msynt B$ and 
		$\thm \msynt A$, then $\thm \msynt B$.
	\end{description}
\end{fact}

\begin{corollary}
	If $\thm \msynt A_1$, \dots, $\thm \msynt A_n$ and
	$\thm \msynt A_1\to \dots\to\msynt A_n\to\msynt B$, then
	$\thm \msynt B$.
\end{corollary}

\begin{lemma}
	If $n\ge 2$ and $\msynt A_1 \lor \dots \lor \msynt A_n$ is a tautology,
	then $\thm \msynt A_1 \lor \dots \lor \msynt A_n$.
\end{lemma}

\begin{fact}
	Suppose that it is associated to each formula \synt A a formula \synt A$^*$
	such that $(\msynt A \lor \msynt B)^* = \msynt A^* \lor \msynt B^*$ and 
	$(\lnot \msynt A)^* = \lnot (\msynt A^*)$. If \synt B is a tautological
	consequence of $\msynt A_1, \dots, \msynt A_n$, then $\msynt B^*$ is a tautological
	consequence of $\msynt A_1^*, \dots, \msynt A_n^*$.
\end{fact}

\begin{fact}[Induction on Theorems]
	To prove that all theorems of $T$ have a property $P$, it suffices showing 
	that:
	\begin{enumerate}
		\item every substitution axiom, identity axiom, equality axiom and nonlogical axiom
		has property $P$,
		\item if $\msynt A_1,\dots,\msynt A_n$ has property $P$ and \synt B is a tautological
		consequence of $\msynt A_1,\dots,\msynt A_n$, then \synt B has the property $P$.
		\item if \synt A has property $P$ and \synt B can be inferred from \synt A
		by the $\exists$-introduction rule, then \synt B has property $P$.
	\end{enumerate}
\end{fact}