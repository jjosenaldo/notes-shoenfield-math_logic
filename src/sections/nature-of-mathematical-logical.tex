\section{The Nature of Mathematical Logic}

\subsection{Formal System}

\begin{definition}[Formal System]
A \emph{formal system} $F$ is roughly the syntactical part of an axiom system. It is composed of:
\begin{enumerate}
    \item A \emph{language}, made up of symbols and formulas. It is denoted as $L(F)$.
    \item \emph{Axioms}, which must be formulas.
    \item \emph{Rules of inference} (or just \emph{rules}), which state that a formula (\emph{conclusion})
    may be inferred from other formulas (\emph{hypotheses}).
\end{enumerate}    
\end{definition}

\begin{definition}[Theorem]
    The set of \emph{theorems} of a formal system is defined by:
    \begin{enumerate}
        \item The axioms are theorems.
        \item If the hypotheses of a rule are theorems, its conclusion is a theorem.
    \end{enumerate}
    We write $\thm_F$ as an abbreviation for \emph{\dots is a theorem of $F$}, and omit 
    the subscript if there is no confusion.
\end{definition}

\begin{definition}[Finite Rule]
    A rule in a formal system is \emph{finite} if it has finitely many hypotheses.
\end{definition}

\begin{fact}
    Inductively defined sets have a ``natural'' associated induction principle.
\end{fact}

\begin{definition}[Proof]
    Let $F$ be a formal system in which all rules are finite. A \emph{proof} in $F$ is a finite sequence
    of formulas, each of which is either an axiom or is the conclusion of a rule whose hypotheses 
    precede that formula in the proof. If $A$ is the last formula in a proof $P$, we say that
    $F$ is a \emph{proof of $A$}. 
\end{definition}

\begin{fact}
    A formula $A$ is a theorem in a formal system $F$ iff there is a proof of $A$ in $F$.
\end{fact}

\begin{definition}[Defined Symbols; Defined Formulas]
    The \emph{defined symbols} of a formal system $F$, which are not a part of $L(F)$, are
    combined with the symbols of $F$ to form expressions called \emph{defined formulas}.
    Defined formulas are abbreviations of formulas of $F$. Also, the length of a defined formula
    is the length of the formula it abbreviates.
\end{definition}

\begin{definition}[Syntactical Variable]
    \emph{Syntactical variables} vary through the expressions of the language under discussion.
    A formula with syntactical variables asserts a scheme of meanings.
\end{definition}

\begin{convention}
    Boldface letters are syntactical variables. In particular, \textbf{u} and \textbf{v}
    vary through all expressions, and \textbf{A}, \textbf{B} and \textbf{C} vary through formulas.
\end{convention}