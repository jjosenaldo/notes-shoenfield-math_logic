\section{Theorems in First-Order Theories}

\subsection{The Tautology Theorem}

\begin{definition}[Elementary Formula]
	A formula is \emph{elementary} if it is either an atomic formula or an
	instantiation.
\end{definition}

\begin{definition}[Truth Valuation]
	A \emph{truth valuation} for a theory $T$ is a mapping 
	from the set of elementary formulas in $T$ to $\set{\mtrue, \mfalse}$.
	Truth valuations are extended to arbitrary formulas in the usual
	manner.
\end{definition}

\begin{definition}[Tautological Consequence]
	$\msynt B$ is a \emph{tautological consequence of} $\msynt A_1, \dots, \msynt A_n$
	if $V(\msynt B)=\mtrue$ for every truth valuation $V$ such that
	$V(\msynt A_1) = \dots = V(\msynt A_n) = \mtrue$.
\end{definition}

\begin{definition}[Tautology]
	A formula is a \emph{tautology} if it is a logical consequence of the
	empty set of formulas.
\end{definition}

\begin{fact}
	$\msynt B$ is a tautological consequence of $\msynt A_1, \dots, \msynt A_n$ iff
	$\msynt A_1 \to \dots \to \msynt A_n \to \msynt B$ is a tautology.
\end{fact}

From now on, formulas are considered to be in the set of formulas whose
logical connectives are either $\lnot$ or $\lor$. 
% Note that they do NOT contain the existential connective

\begin{fact}
	Let $\msynt A_1, \dots, \msynt A_n$ be formulas.
	 %whose logical connectives can only be $\lor$ or $\lnot$. 
	 It can be determined in a finite number of steps
	whether $\msynt A_1 \lor \dots \lor \msynt A_n$ is a tautology.
\end{fact}

\begin{theorem}[Tautology Theorem (Post)]
	If \synt B is a tautological consequence of \synt A$_1$, \dots, 
	\synt A$_n$ and $\thm \msynt A_1$, \dots, $\thm \msynt A_n$,
	then $\thm \msynt B$.
\end{theorem}

\begin{corollary}
	Every tautology is a theorem.
\end{corollary}

\begin{lemma}
	If $\thm \msynt A \lor \msynt B$, then $\thm \msynt B \lor \msynt A$.
\end{lemma}

\begin{fact}
	The detachment rule is admissible: %TODO: what's the right term for this?
	\begin{description}
		\item[Detachment Rule] If $\thm \msynt A \to \msynt B$ and 
		$\thm \msynt A$, then $\thm \msynt B$.
	\end{description}
\end{fact}

\begin{corollary}
	If $\thm \msynt A_1$, \dots, $\thm \msynt A_n$ and
	$\thm \msynt A_1\to \dots\to\msynt A_n\to\msynt B$, then
	$\thm \msynt B$.
\end{corollary}

\begin{lemma}
	If $n\ge 2$ and $\msynt A_1 \lor \dots \lor \msynt A_n$ is a tautology,
	then $\thm \msynt A_1 \lor \dots \lor \msynt A_n$.
\end{lemma}

\begin{fact}
	Suppose that it is associated to each formula \synt A a formula \synt A$^*$
	such that $(\msynt A \lor \msynt B)^* = \msynt A^* \lor \msynt B^*$ and 
	$(\lnot \msynt A)^* = \lnot (\msynt A^*)$. If \synt B is a tautological
	consequence of $\msynt A_1, \dots, \msynt A_n$, then $\msynt B^*$ is a tautological
	consequence of $\msynt A_1^*, \dots, \msynt A_n^*$.
\end{fact}

\begin{fact}[Induction on Theorems]
	To prove that all theorems of $T$ have a property $P$, it suffices showing 
	that:
	\begin{enumerate}
		\item every substitution axiom, identity axiom, equality axiom and nonlogical axiom
		has property $P$,
		\item if $\msynt A_1,\dots,\msynt A_n$ has property $P$ and \synt B is a tautological
		consequence of $\msynt A_1,\dots,\msynt A_n$, then \synt B has the property $P$.
		\item if \synt A has property $P$ and \synt B can be inferred from \synt A
		by the $\exists$-introduction rule, then \synt B has property $P$.
		% Note that the tautology theorem only took care of the formulas w/o existentials
	\end{enumerate}
\end{fact}

\subsection{Results on Quantifiers}

\begin{fact}
	Here are some derivable rules on quantifiers:
	\begin{description}
		\item[$\forall$-Introduction Rule] If $\thm \msynt A \to \msynt B$ and
		\synt x is not free in \synt A, then $\thm \msynt A \to \forall \msynt x \msynt B$.
		\item[Generalization Rule] If $\thm \msynt A$, then $\thm \forall \msynt{xA}$.
		\item[Substitution Rule] If $\thm \msynt A$ and $\msynt A'$ is an instance
		of \synt A, then $\thm \msynt A'$.
		\item[Substitution Theorem] 
		\begin{enumerate}
			\item[]
			\item $\thm \msubs{\msynt a_1,\dots,\msynt a_n}{\msynt x_1,\dots,\msynt x_n}{\msynt A}\to
			\exists \msynt x_1\dots\msynt x_n\msynt A$
			\item $\thm \forall \msynt x_1\dots\forall\msynt x_n\msynt A\to
			\msubs{\msynt a_1,\dots,\msynt a_n}{\msynt x_1,\dots,\msynt x_n}{\msynt A}$	
		\end{enumerate}
		\item[Distribution Rule] If $\thm \msynt A \to \msynt B$, then
		$\thm \exists \msynt{xA}\to\exists\msynt{xB}$ and
		$\thm \forall \msynt{xA}\to\forall\msynt{xB}$
		\item[Closure Theorem] If $\msynt A'$ is the closure of \synt A, then 
		$\thm \msynt A'$ iff $\thm \msynt A$.
	\end{description}
\end{fact}

\begin{corollary}
	If $\msynt A'$ is the closure of \synt A, then \synt A is valid in a
	structure $\struct$ iff $\msynt A'$ is valid in $\struct$.
\end{corollary}

\subsection{The Deduction Theorem}

\begin{theorem}[Deduction Theorem]
	Let \synt A be a closed formula in $T$. For every formula \synt B
	of $T$, $\thm_T \msynt A\to\msynt B$ iff $\thm_{T[\msynt A]}\msynt B$.
\end{theorem}

\begin{corollary}
	Let $\msynt A_1,\dots,\msynt A_n$ be closed formulas in $T$. For every
	formula \synt B in $T$, $\thm_T\msynt A_1\to\dots\to\msynt A_n\to\msynt B$
	iff $\thm_{T[\msynt A_1,\dots,\msynt A_n]}\msynt B$.
\end{corollary}

\begin{theorem}[Theorem on Constants]
	Let $T'$ be obtained from $T$ by adding some new constants. Then for each formula
	\synt A of $T$, and every sequence $\msynt e_1,\dots,\msynt e_n$ of new
	constants, $\thm_T \msynt A$ iff
	 $\thm_{T'} \msubs{\msynt e_1,\dots,\msynt e_n}{}{\msynt A}$.
\end{theorem}

\begin{fact}
	Let \synt A and \synt B be formulas of $T$, and let $\msynt x_1,\dots,\msynt x_n$
	be all the free variables in \synt A. Form $T'$ from $T$ by adding new constants
	$\msynt e_1,\dots,\msynt e_n$. Then
	\[\thm_T \msynt A\to\msynt B \tiff \thm_{T'[\msubs{\msynt e_1,\dots,\msynt e_n}{}{\msynt A}]}
	\msubs{\msynt e_1,\dots,\msynt e_n}{}{\msynt B}\]
\end{fact}